\documentclass[11pt]{report}
\usepackage{outline}
\usepackage{pmgraph}
\usepackage[normalem]{ulem}
\title{\textbf{AerE 361 Lab 14 Report \\ ODE Solver}}
\author{Dalton Michalski}
\date{\oldstylenums{12}/\oldstylenums{7}/\oldstylenums{2018}}
%--------------------Make usable space all of page
\setlength\parindent{24pt}
\setlength{\oddsidemargin}{0in}
\setlength{\evensidemargin}{0in}
\setlength{\topmargin}{0in}
\setlength{\headsep}{-.25in}
\setlength{\textwidth}{6.5in}
\setlength{\textheight}{8.5in}
\usepackage{xcolor}
\definecolor{light-gray}{gray}{0.95}
\newcommand{\code}[1]{\colorbox{light-gray}{\texttt{#1}}}
%--------------------Indention
\setlength{\parindent}{1cm}

\begin{document}
%--------------------Title Page
\maketitle
 
%--------------------Begin Outline
\section{Problem Statement}
\begin{item} A space craft starts in an Earth Geostationary Equitorial Orbit, the spacecraft then fires its engines directly in line with its direction of travel is 90 degrees off from the Earth and the Moon. Using the Runge-Kutta-Fehlberg method to solve differential equations related to the circular restricted 3 body problem, write a program to plot the trajectory of the spacecraft given the change in the velocity of the the spacecraft ($\Delta$V). The initial conditions are:
\begin{equation}
    Y(0) = [ x = 4670.46 (km) \dot{x} = -(9.238 + \Delta V  y = 42164 (km) \dot{y} = 0(km/s)]
\end{equation}
\item When run, the program should 
\begin{enumerate}
    \item Prompt the user for an inital $\Delta$V
    \item Solve the trajectory using RK45
    \item Use GNU plot to display the orbit
    \item Allow user to change $\Delta$V and replot if necessary
    \item When the user is satisfied, generate a high quality .png file
    \item Print out stats about solver
    \item Exit
\end{enumerate}
\end{item}
\newpage
\section{Program Design}
\\ For each of the functions written, the entire \code{struct Round round} was passed as the input.
\begin{enumerate}
    \item \textbf{Calculation:} This function receives one input, a string that represents a mathematical expression. In order to pass this expression to the commandline, the \code{popen} function is used. This functon has been tested and works as expected.
    \item \textbf{Sub-String:} Using the \code{strstr()} function, this function is able to solve the problem and create a pointer to the beginning of the first occurence of that string in a single line of code. This function has been tested inside the simulation and works as expected.
    \item \textbf{Range:} This functions requires two inputs, which can be found at  following memory addresses \begin{enumerate}
        \item \code{int a = round.chal\_dat.range.a}
        \item \code{int b = round.chal\_dat.range.b}
    \end{enumerate}
    The absolute value of (b - a) is assigned to a third integer, and this third integer is also the value returned by this function to the simulation. This program has been tested multiple times and runs as expected.
    \item \textbf{Mean:} This function requires two inputs, an array of numbers to be averaged, as well as the number of datum in the given array. The number of data points can be found at \code{int n = round.chal\_dat.minmaxmean.datalen}. The other data passed to this function is a pointer to the array containing all data. In order to find the mean of this data, a for loop with the form \code{sum += round.chal\_dat.minmaxmean.nums[i]} is used to iterate through all \code{n} data points, During each iteration, the value of each ith data point is added to the running sum. This sum is then divided by \code{n} and returned to the simulation. This function has been heavily tested and works as expected.
    \item \textbf{Minimum:} This function is passed the same two datum as the \textbf{Mean} challenge. In order to find the minimum value, a variables is assigned the value in the first slot of the data array, a for loop is then used to iterate through the remaining data points, each time checking that it isn't smaller than the initial minimum. If it is, the first data point is replaced by the new minimum and the program continues. The for loop continues until all data points have been checked, at this point, the minimum value is then returned to the simulation.
    \item \textbf{Maximum:} This function is identical in every facet to the \textbf{Minimum} fucntion explained above, with the only difference being the greater than sign inside the for loop used to compare all values in the opposite orientation. This function has been tested and works as expected.
    \item \textbf{Reverse:} This function is currently able to correctly reverse the order of the provided string,however, we are yet to be able to write this reversed string to the correct memory location.
    \item \textbf{Find:} This function is passed both a string and a char. In order to return the character index of the given char in the string, and for loop is used to iterate through all chars in the given string, and an if conditon is used to stop the loop when the ith character matches the char being searched for. The function then returns the value of the loop index variable \code{i} to the simulation. This function has been tested and works as expected.
    \item \textbf{Tokenize:} \code{segmentaion fault (core dumped)}
\end{enumerate}
\section{Discussion}
\\The main problems encountered while writing this code were all related to interacting with the provided files. Special care had to be taken in order to ensure each of the challenges was accessing the correct data from the correct structure. Another time consuming portion was trying to figure out how to write the output of all functions to the generic void pointer ans\_ptr as instructed in the manual. In our code, a large switch statement was used under each of the so called "agent" functions. Which case, or function, was called depended on the challenge type passed by the simulation to each respective agent. We were unable to get the "tokenize" function to work correctly, as is was insisten upon returning \code{segmentation fault (core dumped)} despite our best debugging efforts.
\newpage
\section{References}
\\ https://reu.dimacs.rutgers.edu/Symbols.pdf
\\ https://fresh2refresh.com/c-programming/c-function/c-math-h-library-functions/
\\ https://www.overleaf.com/learn/latex/Cross\_referencing\_sections\_and\_equations/
\\ https://www.quora.com/What-is-argc-argv-in-C-What-is-its-purpose-Who-introduced-those-functions
\\ https://www.geeksforgeeks.org/structures-c/
\\ http://steve.hollasch.net/cgindex/coding/ieeefloat.html
\\ https://stackoverflow.com/questions/26284110/strdup-confused-about-warnings-implicit-declaration-makes-pointer-with
\\https://www.w3resource.com/c-programming-exercises/array/c-array-exercise-18.php
\\https://www.quora.com/How-can-I-pass-a-string-as-an-argument-to-a-function-and-return-a-string-in-C
\\https://stackoverflow.com/questions/21488544/triple-pointers-in-c-is-it-a-matter-of-style
\\https://www.quora.com/How-can-I-pass-a-string-as-an-argument-to-a-function-and-return-a-string-in-C
\end{document}